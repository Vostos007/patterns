
\documentclass[12pt]{scrartcl}
\usepackage{fontspec}      % XeLaTeX for robust fonts
\usepackage{polyglossia}   % proper hyphenation and typography per language
\setmainfont{Noto Serif}
\setsansfont{Noto Sans}
\usepackage{graphicx}
\usepackage{caption}
\usepackage{hyperref}
\usepackage{tabularx}
\usepackage{booktabs}
\usepackage{enumitem}
\usepackage{geometry}
\geometry{a4paper, margin=22mm}
\setlength{\parskip}{6pt}
\setlist{nosep}

\usepackage{titlesec}
\titleformat{\section}{\sffamily\Large\bfseries}{}{0pt}{}
\titleformat{\subsection}{\sffamily\large\bfseries}{}{0pt}{}
\titleformat{\subsubsection}{\sffamily\normalsize\bfseries}{}{0pt}{}

% Figure macro with consistent sizing
\newcommand{\KPSFigure}[3]{%
  \begin{figure}[htbp]\centering
  \includegraphics[width=#2]{#1}
  \caption{#3}
  \end{figure}
}

% Cover macro
\newcommand{\KPSCover}[5]{%
  \begin{titlepage}
  \centering
  \vspace*{15mm}
  {\sffamily\bfseries\LARGE #1 \par}
  \vspace{4mm}
  {\sffamily\Large #2 \par}
  \vspace{18mm}
  \includegraphics[width=0.65\textwidth]{#3}\par
  \vfill
  {\small #4 \par}
  {\small #5 \par}
  \end{titlepage}
}

% Table styling helper
\newenvironment{KPSTable}{\small\setlength{\tabcolsep}{6pt}}{}

% Reusable size table
\newcommand{\SizeTable}{%
\begin{KPSTable}
\begin{tabularx}{\textwidth}{l *{4}{>{\centering\arraybackslash}X}}
\toprule
\textbf{Размер} & 6{,}5 & 7 & 7{,}5 & 8 \\
\midrule
\textbf{Обхват кисти (см)} & 18 & 19 & 20{,}5 & 21{,}5 \\
\bottomrule
\end{tabularx}
\end{KPSTable}
}

% Materials table
\newcommand{\MaterialsRU}{%
\begin{itemize}
  \item Casagrande Yak 80 — 1 моток для всех размеров \textit{или} Gruendl Hot Socks Pearl uni — для размеров 6{,}5 и 7 — 1 моток; для 7{,}5 и 8 — 2 мотка
  \item Круговые спицы 2{,}25 мм
  \item Маркеры для вязания
  \item Игла для сшивания
  \item Сантиметровая лента, ножницы
  \item Дополнительная нить
\end{itemize}
}

\newcommand{\MaterialsEN}{%
\begin{itemize}
  \item Casagrande Yak 80 — 1 skein for all sizes \textit{or} Gruendl Hot Socks Pearl uni — sizes 6.5 \& 7: 1 skein; sizes 7.5 \& 8: 2 skeins
  \item 2.25 mm circular needles
  \item Stitch markers
  \item Tapestry needle
  \item Measuring tape, scissors
  \item Waste yarn
\end{itemize}
}

\newcommand{\MaterialsFR}{%
\begin{itemize}
  \item Casagrande Yak 80 — 1 pelote pour toutes les tailles \textit{ou} Gruendl Hot Socks Pearl uni — tailles 6{,}5 \& 7 : 1 pelote ; 7{,}5 \& 8 : 2 pelotes
  \item Aiguilles circulaires 2{,}25 mm
  \item Marqueurs
  \item Aiguille à laine
  \item Mètre ruban, ciseaux
  \item Fil de secours
\end{itemize}
}



\setdefaultlanguage{english}
\title{BONJOUR Gloves}
\author{Hollywool}
\date{}

\begin{document}
\KPSCover{BONJOUR Gloves}{Knitting Pattern}{../images/cover.jpg}{For personal use only. All materials belong to Hollywool.}{Instagram: @hollywool.ru}

\tableofcontents

\section{Materials}
\MaterialsEN

\section{Sizes}
Measure the hand circumference around the widest part across the knuckles with the hand open and fingers relaxed.
\begin{KPSTable}
\begin{tabularx}{\textwidth}{l *{4}{>{\centering\arraybackslash}X}}
\toprule
\textbf{Size} & 6.5 & 7 & 7.5 & 8 \\
\midrule
\textbf{Hand circ. (cm)} & 18 & 19 & 20.5 & 21.5 \\
\bottomrule
\end{tabularx}
\end{KPSTable}

\section{Gauge (after blocking)}
Stockinette stitch: 30 sts and 42 rows = 10×10 cm. Adjust needle size to obtain gauge.

\section{Construction}
Work the gloves in the round from the cuff in stockinette. The instructions are written for the \textit{magic loop} method. When using DPNs, distribute stitches accordingly.

\section{Abbreviations}
\subsection*{Stockinette in the round} knit all stitches every round.\\
\subsection*{1×1 rib} *k1, p1* around.

\section{Make-1 increases}
\subsection{M1R (right-leaning)}
Lift the strand between the stitches from back to front and knit it through the front loop to twist the stitch.
\KPSFigure{../images/fig_prp.png}{0.85\textwidth}{M1R — right-leaning increase (picked up from the bar).}

\subsection{M1L (left-leaning)}
Lift the strand between the stitches from front to back and knit it through the back loop to twist the stitch.
\KPSFigure{../images/fig_prl.png}{0.85\textwidth}{M1L — left-leaning increase (picked up from the bar).}

\section{Cast-on}
Cast on the required stitch count for your size plus 1 st to join in the round. Join and divide stitches equally between two needle tips. Needle 1 holds palm side and thumb; needle 2 holds back of hand.

\section{Cuff}
Work 2 rounds of 1×1 rib, then 40 rounds stockinette. Recommended cuff height: 10 cm.

\section{Thumb gusset}
Increase every other round using M1R/M1L to form the thumb gusset on the palm side. After placing the thumb stitches on hold and casting on bridge stitches, you should have 52/56/60/64 sts to continue to the pinky base.

\end{document}
