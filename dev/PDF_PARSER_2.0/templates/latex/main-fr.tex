
\documentclass[12pt]{scrartcl}
\usepackage{fontspec}      % XeLaTeX for robust fonts
\usepackage{polyglossia}   % proper hyphenation and typography per language
\setmainfont{Noto Serif}
\setsansfont{Noto Sans}
\usepackage{graphicx}
\usepackage{caption}
\usepackage{hyperref}
\usepackage{tabularx}
\usepackage{booktabs}
\usepackage{enumitem}
\usepackage{geometry}
\geometry{a4paper, margin=22mm}
\setlength{\parskip}{6pt}
\setlist{nosep}

\usepackage{titlesec}
\titleformat{\section}{\sffamily\Large\bfseries}{}{0pt}{}
\titleformat{\subsection}{\sffamily\large\bfseries}{}{0pt}{}
\titleformat{\subsubsection}{\sffamily\normalsize\bfseries}{}{0pt}{}

% Figure macro with consistent sizing
\newcommand{\KPSFigure}[3]{%
  \begin{figure}[htbp]\centering
  \includegraphics[width=#2]{#1}
  \caption{#3}
  \end{figure}
}

% Cover macro
\newcommand{\KPSCover}[5]{%
  \begin{titlepage}
  \centering
  \vspace*{15mm}
  {\sffamily\bfseries\LARGE #1 \par}
  \vspace{4mm}
  {\sffamily\Large #2 \par}
  \vspace{18mm}
  \includegraphics[width=0.65\textwidth]{#3}\par
  \vfill
  {\small #4 \par}
  {\small #5 \par}
  \end{titlepage}
}

% Table styling helper
\newenvironment{KPSTable}{\small\setlength{\tabcolsep}{6pt}}{}

% Reusable size table
\newcommand{\SizeTable}{%
\begin{KPSTable}
\begin{tabularx}{\textwidth}{l *{4}{>{\centering\arraybackslash}X}}
\toprule
\textbf{Размер} & 6{,}5 & 7 & 7{,}5 & 8 \\
\midrule
\textbf{Обхват кисти (см)} & 18 & 19 & 20{,}5 & 21{,}5 \\
\bottomrule
\end{tabularx}
\end{KPSTable}
}

% Materials table
\newcommand{\MaterialsRU}{%
\begin{itemize}
  \item Casagrande Yak 80 — 1 моток для всех размеров \textit{или} Gruendl Hot Socks Pearl uni — для размеров 6{,}5 и 7 — 1 моток; для 7{,}5 и 8 — 2 мотка
  \item Круговые спицы 2{,}25 мм
  \item Маркеры для вязания
  \item Игла для сшивания
  \item Сантиметровая лента, ножницы
  \item Дополнительная нить
\end{itemize}
}

\newcommand{\MaterialsEN}{%
\begin{itemize}
  \item Casagrande Yak 80 — 1 skein for all sizes \textit{or} Gruendl Hot Socks Pearl uni — sizes 6.5 \& 7: 1 skein; sizes 7.5 \& 8: 2 skeins
  \item 2.25 mm circular needles
  \item Stitch markers
  \item Tapestry needle
  \item Measuring tape, scissors
  \item Waste yarn
\end{itemize}
}

\newcommand{\MaterialsFR}{%
\begin{itemize}
  \item Casagrande Yak 80 — 1 pelote pour toutes les tailles \textit{ou} Gruendl Hot Socks Pearl uni — tailles 6{,}5 \& 7 : 1 pelote ; 7{,}5 \& 8 : 2 pelotes
  \item Aiguilles circulaires 2{,}25 mm
  \item Marqueurs
  \item Aiguille à laine
  \item Mètre ruban, ciseaux
  \item Fil de secours
\end{itemize}
}



\setdefaultlanguage{french}
\title{Gants BONJOUR}
\author{Hollywool}
\date{}

\begin{document}
\KPSCover{Gants BONJOUR}{Modèle de tricot}{../images/cover.jpg}{Usage personnel uniquement. Tous les contenus appartiennent à Hollywool.}{Instagram : @hollywool.ru}

\tableofcontents

\section{Matériaux}
\MaterialsFR

\section{Tailles}
Mesurez la circonférence de la main autour de la partie la plus large au niveau des articulations, main ouverte et doigts détendus.
\begin{KPSTable}
\begin{tabularx}{\textwidth}{l *{4}{>{\centering\arraybackslash}X}}
\toprule
\textbf{Taille} & 6{,}5 & 7 & 7{,}5 & 8 \\
\midrule
\textbf{Tour de main (cm)} & 18 & 19 & 20{,}5 & 21{,}5 \\
\bottomrule
\end{tabularx}
\end{KPSTable}

\section{Échantillon (après blocage)}
Jersey endroit : 30 m et 42 r = 10×10 cm. Adaptez la taille d’aiguille pour obtenir l’échantillon.

\section{Construction}
Les gants se tricotent en rond depuis le poignet, en jersey. Les explications sont écrites pour la méthode \textit{magic loop}. Avec des aiguilles doubles pointes, répartissez les mailles en conséquence.

\section{Abréviations}
\subsection*{Jersey en rond} toutes les mailles à l'endroit à chaque tour.\\
\subsection*{Côtes 1×1} *1 end, 1 env* sur tout le tour.

\section{Augmentations intercalaires}
\subsection{M1R — inclinée à droite}
Soulevez le brin entre les mailles de l’arrière vers l’avant et tricotez-le torse par le brin avant.
\KPSFigure{../images/fig_prp.png}{0.85\textwidth}{M1R — augmentation intercalaire torsée vers la droite.}

\subsection{M1L — inclinée à gauche}
Soulevez le brin de l’avant vers l’arrière et tricotez-le torse par le brin arrière.
\KPSFigure{../images/fig_prl.png}{0.85\textwidth}{M1L — augmentation intercalaire torsée vers la gauche.}

\section{Montage des mailles}
Montez le nombre de mailles requis pour votre taille, plus 1 m pour fermer en rond. Aiguille 1 : paume et pouce ; Aiguille 2 : dos de la main.

\section{Poignet}
Tricotez 2 tours de côtes 1×1, puis 40 tours en jersey. Hauteur conseillée : 10 cm.

\section{Gousset du pouce}
Augmentez tous les 2 tours (M1R/M1L) pour former le gousset côté paume. Après la mise en attente du pouce et le montage des mailles de pont, il reste 52/56/60/64 m jusqu’à la base de l’auriculaire.

\end{document}
