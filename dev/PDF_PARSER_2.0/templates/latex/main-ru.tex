
\documentclass[12pt]{scrartcl}
\usepackage{fontspec}      % XeLaTeX for robust fonts
\usepackage{polyglossia}   % proper hyphenation and typography per language
\setmainfont{Noto Serif}
\setsansfont{Noto Sans}
\usepackage{graphicx}
\usepackage{caption}
\usepackage{hyperref}
\usepackage{tabularx}
\usepackage{booktabs}
\usepackage{enumitem}
\usepackage{geometry}
\geometry{a4paper, margin=22mm}
\setlength{\parskip}{6pt}
\setlist{nosep}

\usepackage{titlesec}
\titleformat{\section}{\sffamily\Large\bfseries}{}{0pt}{}
\titleformat{\subsection}{\sffamily\large\bfseries}{}{0pt}{}
\titleformat{\subsubsection}{\sffamily\normalsize\bfseries}{}{0pt}{}

% Figure macro with consistent sizing
\newcommand{\KPSFigure}[3]{%
  \begin{figure}[htbp]\centering
  \includegraphics[width=#2]{#1}
  \caption{#3}
  \end{figure}
}

% Cover macro
\newcommand{\KPSCover}[5]{%
  \begin{titlepage}
  \centering
  \vspace*{15mm}
  {\sffamily\bfseries\LARGE #1 \par}
  \vspace{4mm}
  {\sffamily\Large #2 \par}
  \vspace{18mm}
  \includegraphics[width=0.65\textwidth]{#3}\par
  \vfill
  {\small #4 \par}
  {\small #5 \par}
  \end{titlepage}
}

% Table styling helper
\newenvironment{KPSTable}{\small\setlength{\tabcolsep}{6pt}}{}

% Reusable size table
\newcommand{\SizeTable}{%
\begin{KPSTable}
\begin{tabularx}{\textwidth}{l *{4}{>{\centering\arraybackslash}X}}
\toprule
\textbf{Размер} & 6{,}5 & 7 & 7{,}5 & 8 \\
\midrule
\textbf{Обхват кисти (см)} & 18 & 19 & 20{,}5 & 21{,}5 \\
\bottomrule
\end{tabularx}
\end{KPSTable}
}

% Materials table
\newcommand{\MaterialsRU}{%
\begin{itemize}
  \item Casagrande Yak 80 — 1 моток для всех размеров \textit{или} Gruendl Hot Socks Pearl uni — для размеров 6{,}5 и 7 — 1 моток; для 7{,}5 и 8 — 2 мотка
  \item Круговые спицы 2{,}25 мм
  \item Маркеры для вязания
  \item Игла для сшивания
  \item Сантиметровая лента, ножницы
  \item Дополнительная нить
\end{itemize}
}

\newcommand{\MaterialsEN}{%
\begin{itemize}
  \item Casagrande Yak 80 — 1 skein for all sizes \textit{or} Gruendl Hot Socks Pearl uni — sizes 6.5 \& 7: 1 skein; sizes 7.5 \& 8: 2 skeins
  \item 2.25 mm circular needles
  \item Stitch markers
  \item Tapestry needle
  \item Measuring tape, scissors
  \item Waste yarn
\end{itemize}
}

\newcommand{\MaterialsFR}{%
\begin{itemize}
  \item Casagrande Yak 80 — 1 pelote pour toutes les tailles \textit{ou} Gruendl Hot Socks Pearl uni — tailles 6{,}5 \& 7 : 1 pelote ; 7{,}5 \& 8 : 2 pelotes
  \item Aiguilles circulaires 2{,}25 mm
  \item Marqueurs
  \item Aiguille à laine
  \item Mètre ruban, ciseaux
  \item Fil de secours
\end{itemize}
}



\setdefaultlanguage{russian}
\title{Перчатки BONJOUR}
\author{Hollywool}
\date{}

\begin{document}
\KPSCover{Перчатки BONJOUR}{Инструкция по вязанию}{../images/cover.jpg}{Данная инструкция предназначена только для личного использования. Все материалы принадлежат магазину Hollywool.}{Instagram: @hollywool.ru}

\tableofcontents

\section{Материалы}
\MaterialsRU

\section{Размеры}
Для определения обхвата кисти плотно обвейте сантиметровую ленту вокруг самой широкой части руки на уровне костяшек пальцев. Кисть должна быть раскрыта, пальцы расслаблены.
\SizeTable

\section{Плотность (после ВТО)}
Лицевая гладь: 30 петель на 42 ряда = 10×10 см. Спицы необходимо подобрать для получения указанной плотности.

\section{Конструкция}
Перчатки вяжутся по кругу от манжеты лицевой гладью. Ход работы описан для метода \textit{magic loop}. Для вязания на чулочных спицах распределите петли самостоятельно.

\section{Условные обозначения / сокращения}
\subsection*{Лицевая гладь при круговом вязании} лицевые петли в каждом ряду.\\
\subsection*{Резинка 1×1} 1 лицевая, 1 изнаночная по кругу.

\section{Прибавки из протяжки}
\subsection{Прибавка с наклоном вправо (ПрП / M1R)}
Вставьте левую спицу \textit{сзади наперёд} под горизонтальную нить между последней провязанной петлёй и следующей; провяжите эту нить как лицевую за \textit{переднюю} стенку, чтобы нить вкрутилась в петлю.
\KPSFigure{../images/fig_prp.png}{0.85\textwidth}{Прибавка из протяжки вправо (ПрП, M1R).}

\subsection{Прибавка с наклоном влево (ПрЛ / M1L)}
Вставьте левую спицу \textit{спереди назад} под горизонтальную нить между последней провязанной петлёй и первой петлёй на левой спице; провяжите эту нить за \textit{заднюю} стенку, чтобы закрутить её.
\KPSFigure{../images/fig_prl.png}{0.85\textwidth}{Прибавка из протяжки влево (ПрЛ, M1L).}

\section{Набор петель}
Наберите необходимое количество петель для наборного края перчатки вашего размера, плюс 1 петлю для соединения в круг. Соедините вязание в круг и распределите петли пополам на 2 спицы.
Первая спица — внутренняя сторона перчатки и большой палец; вторая — внешняя сторона.

\section{Манжета}
Провяжите 2 ряда резинкой 1×1, далее 40 рядов лицевой гладью. Рекомендованная высота манжеты — 10 см.

\section{Клин большого пальца (с прибавлениями)}
Клин формируется на стороне ладони. Прибавляйте петли через 2 ряда из протяжек. После отделения большого пальца на спицах останется 52/56/60/64 петли и далее вяжите до основания мизинца.

\end{document}
